\usepackage{ucs} 				% Dokument in utf8-Codierung schreiben und speichern
\usepackage{url}
\usepackage{hyperref}
\usepackage[utf8]{inputenc} 	% ermöglicht die direkte Eingabe von Umlauten
\usepackage[english, ngerman]{babel} 	% deutsche Trennungsregeln und Übersetzung der festcodierten Überschriften
\usepackage[T1]{fontenc} 		% Ausgabe aller zeichen in einer T1-Codierung (wichtig für die Ausgabe von Umlauten!)
\usepackage{graphicx}  			% Einbinden von Grafiken erlauben
\usepackage{amsmath}
\usepackage{csquotes}           % Zusatzpaket für Zitierungen (Nur optisch, nicht logisch)
%\usepackage{amsfonts}
%\usepackage{amssymb}
\usepackage{mathpazo} 			% Einstellung der verwendeten Schriftarten
\usepackage{textcomp} 			% zum Einsatz von Eurozeichen u. a. Symbolen
\usepackage{listings}			% Datstellung von Quellcode mit den Umgebungen {lstlisting}, \lstinline und \lstinputlisting
\usepackage[dvipsnames]{xcolor} % einfache Verwendung von Farben in nahezu allen Farbmodellen
\usepackage[intoc]{nomencl} 	% zur Erstellung des Abkürzungsberzeichnisses
\usepackage{fancyhdr}			% Zusatzpaket zur Gestaltung von Fuß und Kopfzeilen
\usepackage{epigraph}
\usepackage{svg}
\usepackage{tkz-kiviat}

\usepackage{xargs}
\usepackage{setspace}
\usepackage[smaller]{acronym}

\usepackage[]{algorithm2e} 		% Für die Verwendung von Pseudocode
\usepackage[]{algorithmicx}		% Anderes Pseudocode Paket zum testen

\usepackage[colorinlistoftodos,prependcaption,textsize=tiny]{todonotes}

\usepackage{hyperref}
\usepackage{ragged2e}

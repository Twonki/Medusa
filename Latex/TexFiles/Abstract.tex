\chapter*{Abstract} %*-Variante sorgt dafür, das Abstract nicht im Inhaltsverzeichnis auftaucht
to be done
~\newline
~\newline
\begin{flushleft}
	\begin{tabular}{lp{11cm}}
		\textbf{title:} & Fooling a TrafficSign-AI \\
		\textbf{authors:}  & \autor \\
		
		%einkommentieren für TH Abgabe
		%\textbf{reviewer TH:} & \betreuerth \\
		%[6ex]%formerly 5ex
	\end{tabular} 
\end{flushleft}


\chapter*{Kurzfassung} 
Diese Arbeit zeigt anhand von mehreren wissenschaftlichen Ansätzen, wie Convolutional Neural Networks zur Erkennung und korrekten Klassifikation von Straßenschildern überlistet werden können.
Im Bereich des Autonomen Fahren wurden mit Neuronalen Netzen die Erkennungsrate drastisch gesteigert, allerdings sind diese immer noch fehleranfällig gegenüber gezielt erzeugten Irrbildern. Selbst ohne Informationen über die unterliegende Architektur (sog. Black-Box Angriffe), sollen Angriffe durch anderweitig erzeugte Irrbilder möglich sein.
Die vorgestellten Verfahren \textit{Degenration}, \textit{Saliency Maps} und \textit{Gradient Ascent} werden erfolgreich angewendet, um mithilfe eines eigenen Neuronalen Netz welches als \textit{Substitute} dient, Irrbilder für Angriffe auf ein unbekanntes Neuronales Netz zu erzeugen.
Ein Angriff gilt als "'erfolgreich'", wenn das Bild mit einer Konfidenz größer als 90\% als Straßenschild erkannt ist, welches ein menschlicher Betrachter nicht als solches erkennen würde.

~\newline
\begin{flushleft}
	\begin{tabular}{lp{11cm}}
		Titel:&  \titel \\ 
		Authoren:&  \autor \\
		%einkommentieren für TH Abgabe
		%Prüfer der Hochschule: &  \betreuerth \\ 
		%[6ex]%formerly 5ex	
	\end{tabular} 
\end{flushleft}
\chapter{Einleitung}
\label{cha:Einleitung}
\setlength{\epigraphwidth}{4in}

\section{Motivation}
\todo{Optische Täuschungen, Irrbilder, Rahmenbedingungen}

\section{Ziel der Arbeit}
\label{sec:ZielDerArbeit}
Ziel dieser Arbeit ist es, Methoden und Herangehensweisen vorzustellen, um die Aufgabenstellung des Informaticup 2019 \todo{Hyperlink!} zu erfüllen: 

Hierbei soll ein neuronales Netz, welches sich hinter einer Webschnittstelle verbirgt und Verkehrsschilder erkennt, erfolgreich \textit{überlistet} werden - es sollen absichtlich Bilder erzeugt werden, welche für den Menschen keine Verkehrsschilder sind aber mit einer Konfidenz von über 90\% als solche erkannt werden. 

Die gefundenen Methoden sollen reproduzierbar sein und in einem Maße flexibel, um beliebig viele Irrbilder zu erzeugen. 

~\newline Der erweiterte Rahmen dieser Arbeit umfasst eine Dokumentation der Methoden sowie Verbesserungen und Schlussfolgerungen aus den Implementierungen zu ziehen. 

Ebenfalls geliefert werden alle Elemente, um die erzielten Ergebnisse zu reproduzieren und variieren. 

~\newline Nicht Ziel dieser Arbeit ist, einen Überblick über neuronale Netze, künstliche Intelligenz oder Bildbearbeitung zu vermitteln. 

~\newline Ebenfalls außerhalb dieser Arbeit liegt eine Auswertung, welche Bilder von einem Menschen als Verkehrschilder erkannt werden. Die Aussagen über solche stützen sich ausschließlich auf die persönliche Einschätzung des Projektteams. 
\section{Aufbau der Arbeit}
Innerhalb dieser Arbeit werden zunächst in Kapitel \ref{cha:TrasiAnalyse} Informationen über die Webschnittstelle gesammelt und aufbereitet. 

Im Abschnitt \ref{sec:TrainingsDaten} werden hierfür zunächst die \textit{German-Trafficsign-Recognition-Benchmark} (Kurz: GTSRB) vorgestellt, welche für das Training der Webschnittstelle verwendet wurden. Dieses Datenset bildet ebenfalls einen zentralen Ausgangspunkt für einige der verfolgten Ansätze.

Anschließend werden in Abschnitt \ref{sec:TrasiModell} die Eigenschaften des Modells zusammengefasst. 
Diese bestehen zum einen aus den offiziellen Angaben der Gesellschaft der Informatiker, zum anderen aus gewonnenen Erkenntnissen. Dieses Kapitel bildet die Grundlage, um die Schnittstelle einzuschätzen. 

~\newline Anschließend werden verschiedene Lösungsansätze vorgestellt, beginnend mit der \textit{Degeneration} in Kapitel \ref{cha:Degeneration}. 

Dieser Ansatz verändert iterativ ein Verkehrsschild, und behält die Änderungen bei, sollte der erzielte Score im akzeptablen Bereich liegen. Mit passender Bildveränderungen erzielen höhere Iterationen unkenntliche Ergebnisse. 

Innerhalb des Kapitels wird zunächst im Abschnitt \ref{sec:DegenerationKonzept} die Idee anhand von Pseudocode weiter erläutert und anschließend in Abschnitt  \ref{sec:DegenerationRemote} die Implementierung für die Webschnittstelle gezeigt. Die Ergebnisse liegen gesondert im Abschnitt \ref{sec:DegenerationErgebnisse} vor. 

Neben der Implementierung für die Webschnittstelle werden zum Abschluss des Kapitels in Abschnitt \ref{sec:DegenerationLokal} noch weitere Verbesserungen für eine lokale Implementierung vorgestellt, welche allerdings nicht für die Webschnittstelle tauglich waren.

~\newline ANPE \todo{Gradient Ascent Fooling Aufbau!}

~\newline ANPE \todo{Saliency Map Aufbau!}

~\newline Abschluss dieser Arbeit bildet im Kapitel \ref{cha:Schluss} ein Fazit über die gefundenen Methoden, sowie ein Ausblick auf weiterführende Arbeiten. 
\section{Verwandte Werke und Primärquellen}
\label{sec:VerwandteWerke}

\section{Rahmenbedingungen des Informaticups}
Die Umsetzung der Implementierung erfolgte innerhalb der webbasierten, interaktiven Entwicklungsumgebung Jupyter Notebook \ref{todo} (in der Version 5.7.4) zusammen mit der objektorientierten höheren Programmiersprache Python \ref{todo} (in der Version 3.6.5).


Jupyter Notebook bietet aufgrund seiner plattformübergreifenden Einsatzmöglichkeit und Kompatibilität zu allen gängigen Webbrowsern eine hohe Flexibilität, was die Darstellung und Ausführung von Python-Code angeht. Darüber hinaus bietet Python eine hohe Verfügbarkeit von Open-Source-Repositories im Bereich Datenverarbeitung, Machine Learning und Deep Learning \ref{todo}. Die Programmiersprache wurde ferner im Rahmen der StackOverflow Befragung 2017 von den befragten Softwareentwicklern zur fünftbeleibtesten Technologie des Jahres 2017 gewählt \ref{todo}. Nicht zuletzt ist Python und die inbegriffenen umfangreichen Standardbibliotheken auf allen gängigen Plattformen, wie beispielsweise Linux, Apple MacOS und Microsoft Windows, kostenlos und in quell- oder binärform verfügbar \ref{todo}.


Als Paketmanager wurde die frei verfügbare Anaconda Distribution in der derzeit aktuellsten Version 2018.12 gewählt, da sie eine vereinfachte Paketinstallation und -verwaltung ermöglicht. Darüber hinaus bietet Anaconda die Möglichkeit Jupyter Notebooks sowie Python und dessen verfügbare Pakete in verschiedenen Entwicklungs- und Testumgebungen isoliert voneinander zu verwalten und zu betreiben \ref{todo}. Schließlich erlaubt “Anaconda Accelerate” den programmatischen Zugriff auf numerische Softwarebibliotheken zur beschleunigten Codeausführung auf Intel Prozessoren sowie NVIDIA Grafikkarten \ref{todo}.


Zur fehlerfreien Ausführung des Codes (Saliency Map Verfahren beziehungsweise Gradient Ascent Verfahren) müssen sowohl Python 3.6.5 als auch folgende Python-Bibliotheken in der wie folgt spezifizierten Version in der zur Laufzeit verwendeten Anaconda Environment vorliegen, wie in Tabelle \ref{tabelle} aufgeschlüsselt.


\begin{table}
	\centering
\begin{tabular}{|l|l|p{10.4cm}|}
	\hline 
	Name & Version & Beschreibung \\ 
	\hline\hline 
	Keras& 2.2.4  & Enthält Funktionen für Deep-Learning Anwendungen [7] \\ 
	\hline 
	Torchvision& 0.2.1 & Enthält Datensätze, Modellarchitekturen und gängige Bildtransformationsoperationen für Computer-Vision Anwendungen [8] \\ 
	\hline 
	OpenCV& 3.4.2  & Enthält Funktionen für echtzeit Computer-Vision Anwendungen [9] \\ 
	\hline 
	NumPy&  1.15.3& Enthält Funktionen zur effizienten Durchführung von Vektor- oder Matrizenberechnungen [10] \\ 
	\hline 
	Requests& 2.18.4 & Enthält Funktionen zur Vereinfachung von HTTP Requests [11] \\ 
	\hline 
	Pillow& 5.2.0 & Enthält Funktionen zum laden, modifizieren und speichern von verschiedenen Bilddateiformaten [12] \\ 
	\hline 
	Matplotlib& 2.2.3 & Enthält Funktionen zum Plotten von Graphen oder Bildern [13] \\ 
	\hline 
	SciPy& 1.1.0  & Enthält wissenschaftliche und technische Funktionen zur Datenverarbeitung [14] \\ 
	\hline 
\end{tabular} 
	\caption{Paketabhängigkeiten der implementierten Software}
\label{tab:parameter}
\end{table}

Um die Voraussetzungen zur benötigten Python Version respektive der erforderlichen Python-Bibliotheken zu erfüllen, muss beim ersten Öffnen des Jupyter Notebooks zum Saliency Map Verfahren beziehungsweise zum Gradient Ascent Verfahren immer der Code zur Rubrik “Managing Anaconda Environment” zuerst ausgeführt werden. Andernfalls kann die korrekte Ausführung von weiteren Teilen des Codes in nachfolgenden Rubriken nicht gewährleistet werden.
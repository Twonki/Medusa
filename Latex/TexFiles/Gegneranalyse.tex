\chapter{Analyse der Webschnittstelle}
\label{cha:TrasiAnalyse}
In das Kapitel kommen die Dinge die wir über die Trasi-AI wissen
\section{Eigenschaften des bereitgestellten Neuronalen Netz}
\label{sec:TrainingsDaten}




33 verschiedene aufgezeichnete klassenlabels (im vergleich GTSRB datensatz 43)

\begin{enumerate}
	\item 
	Aus der aufgabenstellung 64x64x3
	\item 
	bilder aus dem GTSRB [quelle] datensatz
	\item Gekürzte Klassen: Aus der analyse geht die Vermutung hervor, dass nur 33 Klassen unterschieden werden, keine 43 wie im orginal datensatz
	\item Softmax-Ausgabefunktion 
	\item Interpolationsfunktion (vllt mit einem Bild in 3 Interpolationsversionen und jeweiligen Score) 
	\item Overfitting bei Trainingsdaten
	\item unzuverlässigkeit bei nicht-Schildern (z.B. OhmLogo)
\end{enumerate}

\section{Transferierbarkeit von Angriffen auf ein Blackbox Modell}
\label{sec:TrasiModell}
Verwandte arbeiten bestätigen  die Transferierbarkeit von Angriffen, die auf einem "eigenen" neuronalen Netz erzeugt wurden und gegen ein unbekanntes Blackbox modell funktioniern

Einschränkungen? Probleme? Rechtfertigung der Implementation eines eigenen Modells


\section{Implementierung eines eigenen Modells zur Klassifizierung von Straßenschildern (Aphrodite)}
-beschreibung des APhrodite Modells


begründungen für die Designentscheidungen? Strategie/Vorbilder?

\todo{Leonhart}
\newcommandx{\unsure}[2][1=]{\todo[linecolor=red,backgroundcolor=red!25,bordercolor=red,#1]{#2}}
\newcommandx{\change}[2][1=]{\todo[linecolor=blue,backgroundcolor=blue!25,bordercolor=blue,#1]{#2}}
\newcommandx{\info}[2][1=]{\todo[linecolor=OliveGreen,backgroundcolor=OliveGreen!25,bordercolor=OliveGreen,#1]{#2}}
\newcommandx{\improvement}[2][1=]{\todo[linecolor=Plum,backgroundcolor=Plum!25,bordercolor=Plum,#1]{#2}}
\newcommandx{\thiswillnotshow}[2][1=]{\todo[disable,#1]{#2}}

\pgfkeys{
	/kiviatgrad/simplify label/.code={
		\ifx\nv\undefined\else
		\pgfmathparse{Mod(\nv,5)}
		\ifdim\pgfmathresult pt>0pt
		\tikzset{opacity=0}
		\fi
		\fi
	}
}	
% ------------------Setting up environment for codes ---------------

%% R-Markdown
\lstset{ %
	language=R,                     % the language of the code
	basicstyle=\footnotesize,       % the size of the fonts that are used for the code
	numbers=left,                   % where to put the line-numbers
	numberstyle=\tiny\color{gray},  % the style that is used for the line-numbers
	stepnumber=1,                   % the step between two line-numbers. If it's 1, each line
	% will be numbered
	numbersep=5pt,                  % how far the line-numbers are from the code
	backgroundcolor=\color{white},  % choose the background color. You must add \usepackage{color}
	showspaces=false,               % show spaces adding particular underscores
	showstringspaces=false,         % underline spaces within strings
	showtabs=false,                 % show tabs within strings adding particular underscores
	frame=single,                   % adds a frame around the code
	rulecolor=\color{black},        % if not set, the frame-color may be changed on line-breaks within not-black text (e.g. commens (green here))
	tabsize=2,                      % sets default tabsize to 2 spaces
	captionpos=b,                   % sets the caption-position to bottom
	breaklines=true,                % sets automatic line breaking
	breakatwhitespace=false,        % sets if automatic breaks should only happen at whitespace
	title=\lstname,                 % show the filename of files included with \lstinputlisting;
	% also try caption instead of title
	keywordstyle=\color{blue},      % keyword style
	commentstyle=\color{dkgreen},   % comment style
	stringstyle=\color{red},      % string literal style
	escapeinside={\%*}{*)},         % if you want to add a comment within your code
%	morekeywords={*,...}            % if you want to add more keywords to the set
} 

\usetikzlibrary{arrows} % define style of tkiz kiviat
\renewcommand{\lstlistingname}{Code}

% Abkürzungen
\newcommand{\ua}{\mbox{u.\,a.\ }}
\newcommand{\zB}{\mbox{z.\,B.\ }}
\newcommand{\bs}{$\backslash$}

\renewcommand{\nomname}{Abkürzungsverzeichnis}

% -------------------------------------------------------------------------------------------
% Definition der Kopf- und Fußzeilen
\lhead{}								% Kopf links
\chead{}								% Kopf mitte
\rhead{\sffamily{\leftmark}}				% Kopf rechts
\lfoot{}								% Fuß links
\cfoot{\sffamily{\thepage}}				% Fuß mitte
\rfoot{\sffamily{\autor}}				% Fuß rechts
\renewcommand{\headrulewidth}{0.4pt}	% Liniendicke Kopf
\renewcommand{\footrulewidth}{0.4pt}	% Liniendicke Fuß


\makenomenclature							% Abkürzungsverzeichnis erstellen
%\input{Inhalt/abkuerzungen}					% Datei mit Abkürzungen laden

% Definition of colors
\definecolor{lightblue}{rgb}{0.910,0.933,0.970}
\definecolor{lightred}{RGB}{247,238,232}
\definecolor{monochromeLightblue}{RGB}{165,188,222}
\definecolor{monochromeLightred}{RGB}{222,188,165}
\definecolor{kiviatOne}{RGB}{137,193,30}
\definecolor{kiviatTwo}{RGB}{20,128,120}
\definecolor{kiviatThree}{RGB}{208,108,32}
\definecolor{kiviatFour}{RGB}{168,26,104}

\newcommand\ColorBox[1]{\textcolor{#1}{\rule{2.5ex}{2.5ex}}}

% ----------------------------------- Links Styling -----------------------------------------
\hypersetup{
	%pdfborder = {0 0 0},
	bookmarks=true,         % show bookmarks bar?
	unicode=false,          % non-Latin characters in Acrobat’s bookmarks
	pdftitle={Erstellung von Irrbildern zur Überlistung einer Verkehrsschilder erkennenden KI}
	pdfauthor={Leonhard Applis, Peter Bauer, Andreas Porada und Florian Stöckl},     % author
	pdfsubject={IT-Projektarbeit},   % subject of the document
	pdfcreator={Leonhard Applis, Peter Bauer, Andreas Porada und Florian Stöckl},   % creator of the document
	pdfproducer={Leonhard Applis, Peter Bauer, Andreas Porada und Florian Stöckl}, % producer of the document
	colorlinks=true,       % false: boxed links; true: colored links
	linkcolor=blue,          % color of internal links (change box color with inkbordercolor)
	citecolor=CadetBlue,        % color of links to bibliography
	filecolor=magenta,      % color of file links
	urlcolor=cyan,           % color of external links
}
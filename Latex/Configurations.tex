\newcommandx{\unsure}[2][1=]{\todo[linecolor=red,backgroundcolor=red!25,bordercolor=red,#1]{#2}}
\newcommandx{\change}[2][1=]{\todo[linecolor=blue,backgroundcolor=blue!25,bordercolor=blue,#1]{#2}}
\newcommandx{\info}[2][1=]{\todo[linecolor=OliveGreen,backgroundcolor=OliveGreen!25,bordercolor=OliveGreen,#1]{#2}}
\newcommandx{\improvement}[2][1=]{\todo[linecolor=Plum,backgroundcolor=Plum!25,bordercolor=Plum,#1]{#2}}
\newcommandx{\thiswillnotshow}[2][1=]{\todo[disable,#1]{#2}}

% Default fixed font does not support bold face
\DeclareFixedFont{\ttb}{T1}{txtt}{bx}{n}{12} % for bold
\DeclareFixedFont{\ttm}{T1}{txtt}{m}{n}{12}  % for normal

%------------Colors ----------------
\definecolor{deepblue}{rgb}{0,0,0.5}
\definecolor{deepred}{rgb}{0.6,0,0}
\definecolor{deepgreen}{rgb}{0,0.5,0}

\pgfkeys{
	/kiviatgrad/simplify label/.code={
		\ifx\nv\undefined\else
		\pgfmathparse{Mod(\nv,5)}
		\ifdim\pgfmathresult pt>0pt
		\tikzset{opacity=0}
		\fi
		\fi
	}
}	
% ------------------Setting up environment for codes ---------------
% Python style for highlighting
\newcommand\pythonstyle{\lstset{
		language=Python,
		numbers=left,			% Zeilennummern links
		stepnumber=1,			% Jede Zeile nummerieren.
		numbersep=5pt,			% 5pt Abstand zum Quellcode
		breaklines=true,		% Zeilen umbrechen wenn notwendig.
		breakautoindent=true,	% Nach dem Zeilenumbruch Zeile einrücken.
		postbreak=\space,		% Bei Leerzeichen umbrechen.
		tabsize=2,				% Tabulatorgrösse 2
		basicstyle=\ttfamily\footnotesize, % Nichtproportionale Schrift, klein für den Quellcode
		showspaces=false,		% Leerzeichen nicht anzeigen.
		showstringspaces=false,	% Leerzeichen auch in Strings ('') nicht anzeigen.
		extendedchars=true,		% Alle Zeichen vom Latin1 Zeichensatz anzeigen.
		captionpos=b,			% sets the caption-position to bottom
		xleftmargin=0pt,		% Rand links
		xrightmargin=0pt,		% Rand rechts
		frame=single,			% Rahmen an
		frameround=ffff,
		rulecolor=\color{darkgray},	% Rahmenfarbe
		basicstyle=\ttm,
		morekeywords={access,and,break,class,continue,def,del,elif,else,except,exec,finally,for,from,global,if,import,in,is,lambda,not,or,pass,print,raise,return,try,while},
		morekeywords=[2]{abs,all,any,basestring,bin,bool,bytearray,callable,chr,classmethod,cmp,compile,complex,delattr,dict,dir,divmod,enumerate,eval,execfile,file,filter,float,format,frozenset,getattr,globals,hasattr,hash,help,hex,id,input,int,isinstance,issubclass,iter,len,list,locals,long,map,max,memoryview,min,next,object,oct,open,ord,pow,property,range,raw_input,reduce,reload,repr,reversed,round,set,setattr,slice,sorted,staticmethod,str,sum,super,tuple,type,unichr,unicode,vars,xrange,zip,apply,buffer,coerce,intern},%
		morecomment=[l]\#,%
		sensitive=true,%
		%otherkeywords={self,if,print,while,return,def},             % Add keywords here
		keywordstyle=\ttb\color{deepblue},
		emph={MyClass,__init__},          % Custom highlighting
		emphstyle=\ttb\color{deepred},    % Custom highlighting style
		stringstyle=\color{deepgreen},
		% Any extra options here
		showstringspaces=false            % 
}}

% Python environment
\lstnewenvironment{python}[1][]
{
	\pythonstyle
	\lstset{#1}
}
{}

% Python for external files
\newcommand\pythonexternal[2][]{{
		\pythonstyle
		\lstinputlisting[#1]{#2}}}


\usetikzlibrary{arrows} % define style of tkiz kiviat

% Abkürzungen
\newcommand{\ua}{\mbox{u.\,a.\ }}
\newcommand{\zB}{\mbox{z.\,B.\ }}
\newcommand{\bs}{$\backslash$}

\renewcommand{\nomname}{Abkürzungsverzeichnis}

% -------------------------------------------------------------------------------------------
% Definition der Kopf- und Fußzeilen
\lhead{}								% Kopf links
\chead{}								% Kopf mitte
\rhead{\sffamily{\leftmark}}				% Kopf rechts
\cfoot{}								% Fuß links
\lfoot{\sffamily{Seite \thepage}}				% Fuß mitte
\rfoot{\sffamily{\autorShort}}				% Fuß rechts
\renewcommand{\headrulewidth}{0.4pt}	% Liniendicke Kopf
\renewcommand{\footrulewidth}{0.4pt}	% Liniendicke Fuß


\makenomenclature							% Abkürzungsverzeichnis erstellen
%\input{Inhalt/abkuerzungen}					% Datei mit Abkürzungen laden

% Definition of colors
\definecolor{lightblue}{rgb}{0.910,0.933,0.970}
\definecolor{lightred}{RGB}{247,238,232}
\definecolor{monochromeLightblue}{RGB}{165,188,222}
\definecolor{monochromeLightred}{RGB}{222,188,165}
\definecolor{kiviatOne}{RGB}{137,193,30}
\definecolor{kiviatTwo}{RGB}{20,128,120}
\definecolor{kiviatThree}{RGB}{208,108,32}
\definecolor{kiviatFour}{RGB}{168,26,104}

\newcommand\ColorBox[1]{\textcolor{#1}{\rule{2.5ex}{2.5ex}}}

% ----------------------------------- Links Styling -----------------------------------------
\hypersetup{
	%pdfborder = {0 0 0},
	bookmarks=true,         % show bookmarks bar?
	unicode=false,          % non-Latin characters in Acrobat’s bookmarks
	pdftitle={\titel}
	pdfauthor={\autor},     % author
	pdfsubject={\ausarbeitung},   % subject of the document
	pdfcreator={\autor},   % creator of the document
	pdfproducer={\autor}, % producer of the document
	colorlinks=true,       % false: boxed links; true: colored links
	linkcolor=blue,          % color of internal links (change box color with inkbordercolor)
	citecolor=CadetBlue,        % color of links to bibliography
	filecolor=magenta,      % color of file links
	urlcolor=cyan,           % color of external links
}